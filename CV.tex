\documentclass[a4paper,french,10pt]{moderncv}
\moderncvstyle{classic}
\moderncvcolor{blue}
\moderncvicons{awesome}

\usepackage{babel}
\frenchsetup{ItemLabelii=\textendash}

\usepackage[default]{opensans}

\usepackage[scale=0.85]{geometry}
\setlength{\hintscolumnwidth}{2.6cm}
\recomputelengths{}

% Personal data
\name{Arnaud}{Rocher}
\title{Automaticien}
\address{38 rue Platon}{72000 Le Mans}
\phone[mobile]{06 79 93 01 17}
\email{arnaud.roche3@gmail.com}
\social[github]{cailloumajor}
\extrainfo{34 ans --- Permis B}

\begin{document}
\maketitle{}

\section{Expérience professionnelle}
\cventry{2013--auj.}%
{Automaticien}{NTN Transmissions Europe}{Allonnes (72)}{}{%
  \begin{itemize}
    \item Amélioration et assistance au dépannage de moyens d'usinage et
          traitement thermique sur les sites d'Allonnes et Sibiu (Roumanie);
    \item Participation aux projets d'acquisition de moyens neufs, pour les
          sites d'Allonnes et Sibiu:
          \begin{itemize}
            \item définition du besoin, rédaction du cahier des charges;
            \item étude et choix d'offres techniques;
            \item suivi au cours de la construction (validations statique et
                  dynamique chez le fournisseur);
            \item suivi du démarrage et réception en fin de mise en route.
          \end{itemize}
  \end{itemize}
}
\cventry{2011--2013}%
{Technicien de maintenance}{NTN Transmissions Europe}{Allonnes (72)}{}{%
  \begin{itemize}
    \item Suivi des pannes longues (extraction GMAO, analyse avec méthode
          Maxer);
    \item Modifications d'automatisme sur moyens de tournage et rectification.
  \end{itemize}
}
\cventry{2005--2011}%
{Dépanneur électro-mécanicien}{NTN Transmissions Europe}{Allonnes (72)}{}{%
  \begin{itemize}
    \item Maintenance préventive et dépannage (dont 5 ans en équipe de nuit)
          sur moyens d'usinage, de traitement thermique et de sous-assemblage.
  \end{itemize}
}
\cventry{2003--2005}%
{Apprenti}{Getelectric}{Grenoble}{}{%
  \begin{itemize}
    \item Étude et réalisation d'installations électriques tertiaires.
  \end{itemize}
}
\cventry{2001--2003}%
{Apprenti}{Auto Châssis International (ex-Renault)}{Le Mans}{}{%
  \begin{itemize}
    \item Maintenance préventive et curative sur moyens d'usinage et
          d'assemblage.
  \end{itemize}
}

\section{Formation}
\cventry{2003--2005}%
{B.T.S. Électrotechnique}{C.F.A.I. Maison de la production}{Moirans (38)
}%
{\textit{14,9 / 20}}{}
\cventry{2001--2003}%
{Bac Pro Équipements et Installations Électriques}{C.F.A.I. de l'A.F.P.}{Le Mans}%
{\textit{17,06 / 20}}{}
\cventry{1999--2001}%
{B.E.P. Électrotechnique}{E.T.P. de l'A.F.P.}{Le Mans}%
{\textit{17,28 / 20}}{}

\section{Compétences}
\subsection{Automatisme / Robotique}
\cvitem{Siemens}{Automates S7--300/1200/1500 (ateliers logiciels STEP7 V5.x et
  TIA Portal), commandes numériques 840D pl/sl, sécurité programmée (S7
  Distributed Safety, STEP7 Safety, SINUMERIK Safety Integrated SPL), réseaux
  PROFIBUS et PROFINET, IHM (logiciels ProTool, WinCC, HMI-PRO).}
\cvitem{KUKA}{Contrôleurs robots KRC1, KRC2, et KRC4.}  % chktex 19
\cvitem{Autres}{Automates Mitsubishi A/Q, Robots et CN Fanuc.}

\subsection{Informatique}
\cvitem{Administration}{Systèmes Linux (Debian), réseau Ethernet, virtualisation.}
\cvitem{Programmation}{Python, HTML, CSS, Javascript, C/C++, informatique
  embarquée, \LaTeX.}

\subsection{Langues}
\cvitem{Anglais}{Lu, parlé, écrit (technique).}

\section{Centres d'intérêt}
\cvitem{}{Informatique, sports mécaniques, langues étrangères (Italien, Roumain).}

\end{document}
